\chapter{Further tools}
\label{sec:tools}

Besides {\sl uvspec} and {\sl mie} {\sl libRadtran} provides several
small tools related to radiative transfer in the atmosphere. These
tools can be found in the \file{bin} directory. 
Some of the tools are described in this chapter.
 
Help for all tools can be obtained on the command line using the
option \code{-h}. 

\section{General tools}

\subsection{Integration - \code{integrate}}
\index{integrate}

\code{integrate} calculates the integral between limits
$x_\mathrm{min}$ and $x_\mathrm{max}$ 
by interpolating the data points (x[i], y[i]) with natural cubic
splines or linear interpolation. $x_\mathrm{min}$ and $x_\mathrm{max}$ are the minimum
and maximum values of the first column in the input file. The 
x-values in the first column must be in ascending order.

The different options to \code{integrate} are displayed when executing:

\begin{verbatim}
integrate -h
\end{verbatim}

\subsection{Interpolation - \code{spline}}
\index{spline}
\index{Interpolation}

\code{spline} interpolates discrete data points using natural cubic
splines or linear interpolation. The x-values in the first column must 
be in ascending order. 

The different options to \code{spline} are displayed when executing:

\begin{verbatim}
spline -h
\end{verbatim}

\index{conv}
\subsection{Convolution - \code{conv}}
                  
\code{conv} convolutes a spectrum with a given filter function.

The different options to \code{conv} are displayed when executing:

\begin{verbatim}
conv -h
\end{verbatim}


\index{addlevel}
\subsection{Add level to profile - \code{addlevel}}

\code{addlevel} is a simple shell script to add a level to 
an atmosphere\_file or mol\_file. It is located in the \code{src} directory.

The different options to \code{addlevel} are displayed 
when executing:

\begin{verbatim}
addlevel -h
\end{verbatim}

\subsection{Numerical difference between two files -\codeidx{ndiff}}
The Perl script \code{ndiff} calculates the relative difference between two files
containing columns of numbers (file1/file0). The first column is
not included. The calculated differences are output to stdout.
If limit is different from 0.0, the number of differences greater
than abs(maxdiff) are printed to stdout. The \code{ndiff} script is
extensively used by the \file{test/test.pl} script invoked by \code{make check}.

The \code{ndiff} script is invoked by
  \begin{Verbatim}[fontsize=\footnotesize]
      ndiff [options] file0 file1
  \end{Verbatim}
The script understands the following options
\begin{description}
   \item[--limit $<$value$>$]     The minimum value in file0 considered when
                                  counting the number of differences between file0
                                  and file1. Default is 0.0.
   \item[--maxdiff $<$value$>$]   The maximum relative difference allowed between
                                  file0 and file1. Defaut is 0.0.
   \item[--sub]                   Subtract file1 - file0 instead of division
   \item[--nox]                   First column is included
   \item[--quiet]                 The differences are not output, but the number of
                                  differences are still printed.
   \item[--help]                  Print help message.
\end{description}


\section{Tools to generate input data to and analyse output data from \code{uvspec}}

% The snowalbedo tool
\input{snowalbedo}


\index{cldprp}
\subsection{Calculate cloud properties - \code{cldprp}}

\code{cldprp} calculates wavelength-dependent cloud properties 
using one of several parameterizations.

The different options to \code{cldprp} are displayed when executing:

\begin{verbatim}
cldprp -h
\end{verbatim}

% CE: Commented cldgen since it is not clear whether it is working or not !!!

% \index{cldgen}
% \subsection{Generate model clouds - \code{cldgen}}

% \code{cldgen} generates 1D, 2D, and 3D model clouds to be used as 
% input to libRadtran / MYSTIC.

% \begin{verbatim}
% cldgen < input_file > output_file
% \end{verbatim}

% format of the input and output files are described
% below. Several realistic examples of input files are subsequently 
% given.

% \strong{Warning:} Please note the error checking on input variables 
% is very scarce at the moment. Hence, if you provide erroneous input, 
% the outcome is unpredictable.
   

% \subsubsection{The \code{cldgen} input file}

% The cldgen input file consists of single line entries, each making
% up a complete input to the cldgen program. First on the line comes
% the parameter name, followed by one or more parameter values. The 
% parameter name and the parameter values are seperated by white space.

% Filenames are entered without any surrounding single or double quotes.

% Comments are introduced by a \code{\#}. Blank lines are ignored.

% The various input parameters are described in detail below.

% \FIXME{the following should be extracted from cldgen\_lex.l}


% \subsubsection{The \code{cldgen} output}

% \subsubsection{Example}

% An example of a complete input file is

% \efile{../examples/CLDGEN.INP}

% The zenith tool
\input{zenith}

% The noon tool
\input{noon}

% The angres tool
\input{angres}

% The make_angresfunc tool
\input{make_angresfunc}

% The make_slitfunction tool
\input{make_slitfunction}

% Calculate phase function
\input{phase}

% Legendre decomposition
\input{pmom}

% automated independent pixel calculation
% \input{worldloop}

\section{Other useful tools}




\input{read_Stamnes_tab}

  
