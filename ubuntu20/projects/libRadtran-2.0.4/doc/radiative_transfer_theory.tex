\chapter{Radiative transfer theory}
\label{chap:rttheory}

\section{Overview}

Radiative transfer in planetary atmospheres is a complex problem. The
best tool for the solution may vary depending on the problem.
The {\sl libRadtran} package contains numerous tools that handle
various aspects of atmospheric radiative transfer. The main tools
will be presented later in chapter~\ref{chap:rt}. To give the
user a background for the problem to be solved, the theory behind will
briefly be presented below. The radiative transfer
equation is presented first, and solution methods and approximations are 
outlined afterwards. 

The number of equations in this chapter may be intimidating
even for the brave-hearted. If you just want to get things done and
wonder if the {\sl libRadtran} package includes tools that may be used
for your problem, jump directly to chapter~\ref{chap:rt}. Another good
starting point is to try the examples available through the Graphical
User Interface to the \code{uvspec} tool.


\section{The radiative transfer equation}
Quite generally, the distribution of photons in a dilute gas may be
described by the Boltzmann equation\footnote{For a derivation of the
  Boltzmann equation see a textbook on statistical mechanics, for
  example \citet{Reif1965}. Also note that the Boltzmann equation is
  not a fundamental equation. For a derivation of the radiative
  transfer equation from the Maxwell equations see
  \citet{Mishchenko2002}.} 
\begin{eqnarray}
  {\frac{\partial f}{\partial t}} + \nabla_{{\bf r}}({\bf v}\; f ) +
  \nabla_{{\bf p}}({\bf F} \;f ) = Q({\bf r},\hat{n}, \nu, t).
  \label{eq:Boltzmann}
\end{eqnarray}
Here, the photon distribution function $f({\bf r},\hat{n}, \nu,  t)$ 
varies with location (${\bf r}$), direction of propagation
($\hat{n}$), frequency ($\nu$) and time ($t$). It is defined such that
\begin{eqnarray}
f({\bf r},\hat{n}, \nu, t)\;c\; \hat{n}\cdot d{\bf S}\;d\Omega\; d\nu\; dt
  \label{eq:photon_distribution}
\end{eqnarray}
represents the number of photons with frequency between $\nu$ and $\nu+d\nu$
crossing a surface element $d{\bf S}$ in direction $\hat{n}$ into
solid angle $d\Omega$ in time $dt$ (Stamnes 1986). 
The units of $f({\bf r},\hat{n}, \nu, t)$ are $cm^{-3}\; sr^{-1}\; Hz^{-1}$
and $c$ is the speed of light.
Furthermore, $\nabla_{{\bf r}}$ and $\nabla_{{\bf p}}$ are the divergence
operators in configuration and momentum space, respectively. The 
photons may be subject to an external force ${\bf F}({\bf r},\hat{n}, \nu, t)$
and there may be sources and sinks of photons due to collisions and/or
`true' production and loss, which are represented by 
$Q({\bf r},\hat{n}, \nu, t)$. 

In the absence of relativistic effects ${\bf F}=0$, and the photons propagate 
in straight lines with velocity ${\bf v}=c\;\hat{n}$ between
collisions. Using the relation  
\begin{eqnarray}
\nabla_{{\bf r}}({\bf v}\; f ) = f\;\nabla_{{\bf r}}{\bf v} +
{\bf v}\cdot \nabla f  = {\bf v}\cdot \nabla f,
\end{eqnarray}
where {\bf r} and {\bf v} are independent variables,
Eq.~\ref{eq:Boltzmann} may be written as  
\begin{eqnarray}
  {\partial f \over \partial t} + c\;( \hat{n}\cdot \nabla)\; f
  = Q({\bf r},\hat{n}, \nu, t)
\label{eq:Boltzmann_simpler}
\end{eqnarray}
where the ${\bf r}$ subscript on the gradient operator $\nabla$  has
been omitted. 

The differential energy associated with the photon distribution is
\begin{eqnarray}
dE = c\; h\nu\; f\; \hat{n}\cdot d{\bf S} \;d\Omega\; d\nu \; dt.
  \label{eq:diff_energy}
\end{eqnarray}
The specific intensity of photons
$I({\bf r},\hat{n}, \nu, t)$ is defined such that
($\hat{n}\cdot d{\bf S} = \cos \theta \;dS$)
\begin{eqnarray}
dE = I({\bf r},\hat{n}, \nu, t)\; dS \;\cos\theta\; d\Omega\; d\nu \; dt,
  \label{eq2.5}
\end{eqnarray}
which  gives
\begin{eqnarray}
I({\bf r},\hat{n}, \nu, t) = c\; h\nu \;f({\bf r},\hat{n}, \nu, t).
  \label{eq2.6}
\end{eqnarray}
In a steady state situation Eq.~\ref{eq:Boltzmann_simpler} may then be
written as
\begin{eqnarray}
  ( \hat{n}\cdot \nabla)\;I({\bf r},\hat{n}, \nu)  
     = h\nu \;Q({\bf r},\hat{n}, \nu). 
  \label{eq:Boltzmann_rte}
\end{eqnarray}
Eq.~\ref{eq:Boltzmann_rte} may be interpreted as the radiative
transfer equation in a general geometry. However, as long as the
source term $Q({\bf r},\hat{n}, \nu)$ is not specified it is of little
use. First, however, the two most common geometries for radiative
transfer in planetary atmospheres will be described.

\subsection{The streaming term}
The streaming term $\hat{n}\cdot \nabla$ defines the geometry. In
planetary atmospheres the cartesian and spherical geometries are
most common. In cartesian geometry the plane-parallel approximation is
often used while in spherical geometry the pseudo-spherical and
spherical shell approximations are popular.
\subsubsection{Cartesian geometry - plane-parallel atmosphere}
In a Cartesian coordinate system the streaming term may be written
\citep{Rottmann1991,Kuo1996}
\begin{eqnarray}
   \hat{n}\cdot \nabla &=& n_x \frac{\partial}{\partial x} + n_y
   \frac{\partial}{\partial y} + n_z \frac{\partial}{\partial
     z}\nonumber \\ &=&
   \cos\phi\sqrt{1-\mu^2}\frac{\partial}{\partial x} +
   \sin\phi\sqrt{1-\mu^2}\frac{\partial}{\partial y} +
   \mu \frac{\partial}{\partial z},
  \label{eq:CartesianStreamingTerm}
\end{eqnarray}
where $(n_x, n_y, n_z)$ are the components of the unit vector,
$\mu=\cos\theta$ and $\phi$ is the azimuth angle.

In a plane-parallel geometry (Flat Earth approximation) the atmosphere
is divided into parallel layers of infinite extensions in the $x$- and
$y$-directions. This implies that there are no variation in the $x$- and
$y$-directions. Hence, for this approximation the streaming term becomes
\begin{eqnarray}
   \hat{n}\cdot \nabla =  \mu \frac{\partial}{\partial z}.
  \label{eq:PlaneParallelStreamingTerm}
\end{eqnarray}
This approximation is used by numerous radiative transfer solvers,
including the much used DISORT solver \citep{Stamnes1988c}.

\subsubsection{Spherical geometry - pseudo-spherical atmosphere}
In spherical geometry the streaming term becomes\footnote{A derivation
is provided in Appendix~O of \citet{Thomas1999}. The appendix is
available from http://odin.mat.stevens-tech.edu/rttext/.} 
\begin{eqnarray}
   \hat{n}\cdot{\bf\nabla} &=& 
\mu\frac{\partial}{\partial r} + 
\frac{1-\mu^{2}}{r}\frac{\partial}{\partial \mu}  \nonumber \\
&&\!\!\!\!\!\!\!\!\!\!\!\!\!\!\!\!\!\!\!\!\!\!\!\!\!\!\!
+ \frac{\sqrt{1-\mu^{2}}\sqrt{1-{\mu_{0}}^{2}}}{r}\left[
\cos(\phi-\phi_{0})\frac{\partial}{\partial \mu_{0}} +
\frac{\mu_{0}}{1-\mu_{0}^{2}}\sin(\phi-\phi_{0})
\frac{\partial}{\partial (\phi-\phi_{0})} 
\right].
  \label{eq:SphericalStreamingTerm}
\end{eqnarray}

In a spherically symmetric (=spherical shell) atmosphere the streaming term reduces to
\begin{eqnarray}
   \hat{n}\cdot{\bf\nabla} &=& 
\mu\frac{\partial}{\partial r} + 
\frac{1-\mu^{2}}{r}\frac{\partial}{\partial \mu}.
  \label{eq:SphericalSymmetricStreamingTerm}
\end{eqnarray}
\citet{Dahlback1991} has shown that for mean intensities it is
sufficient to include only the first term in
Eq.~\ref{eq:SphericalSymmetricStreamingTerm} for solar zenith angles
up to 90$^{\circ}$. Thus, 
\begin{eqnarray}
   \hat{n}\cdot{\bf\nabla} &=& 
\mu\frac{\partial}{\partial r}.
  \label{eq:PseudoSphericalSymmetricStreamingTerm}
\end{eqnarray}
For this to hold the direct beam must be calculated in spherical
geometry. This is the so-called pseudo-spherical approximation. It
may work well for irradiances, mean intensities and nadir and zenith
radiances. For irradiances in off-zenith and off-nadir directions it
must be shown the angle derivatives are indeed negligible. This is
rarely done in practice.

\subsection{The source term}
The source term on the right hand side of Eq.~\ref{eq:Boltzmann_rte}
includes all losses and gains of radiation in the direction and
frequency of interest. For photons in a planetary atmosphere the
source term may be written as\footnote{For a derivation of the
  individual terms see e.g. \citet{chand60}.}
\begin{eqnarray}
h\nu\;Q( r,\hat{n}, \nu) &=& h\nu\;Q(r, \theta, \phi, \nu)
= -\beta^{ext}(r, \nu)\;
I(r,\theta, \phi, \nu)
\nonumber\\ &&
+\frac{1}{4\pi}\int_{0}^{\infty}\beta^{sca}(r,\nu,\nu')\int_{0}^{2\pi}d\phi '\int_{0}^{\pi}d\theta '
p(r,\theta, \phi; \theta', \phi') 
I( r,\theta ', \phi ', \nu') d\nu'
\nonumber\\ &&
+\beta^{abs}(r, \nu)\;B[T(r)].
  \label{eq:GeneralSource}
\end{eqnarray}
The first term represents loss of radiation due to absorption and
scattering (=extinction)
out of the photon beam. The second term (multiple scattering term) describes the
number of photons scattered into the beam from all other directions
and frequencies, finally, the third term gives the amount of thermal
radiation emitted in the frequency range of interest. 

The lower part of the Earth's atmosphere, may to a good approximation,
be assumed to be in local thermodynamic equilibrium
\footnote{
The hypothesis of local thermodynamic equilibrium (LTE) makes the
assumption that all thermodynamic properties of the medium are the same as
their thermodynamic equilibrium (T.E.) values at the local $T$ and density.
Only the radiation field is allowed to depart from its T.E. value of
$B[T(r)]$ and is obtained from a solution of the transfer equation. Such
an approach is manifestly {\it internally inconsistent}. $\ldots$ 
`However, if the medium is subject  only to {\it small} gradients over the
mean free path a photon can travel before it is destroyed and 
thermalized by a collisional process, then the LTE approach is valid.'
(adapted from \citet[p. 26]{Mihalas1978})}.
Thus, the
emitted radiation is proportional to the Planck function, $B[T(r)]$,
integrated over the frequency or wavelength region of interest.
Furthermore, by Kirchhoff's law the emissivity coefficient $\beta^{emi}$
is equal to the absorption coefficient $\beta^{abs}$.

The absorption, scattering and extinction coefficients are defined as
\citep{Stamnes1986b}
\begin{eqnarray}
\beta^{abs}(r, \nu)=\sum_{i}\beta^{abs}_{i}(r, \nu), &&
\beta^{abs}_{i}(r, \nu)=n_{i}(r) \sigma_{i}^{abs}(\nu)
\label{eq:abs}
\\
\beta^{sca}(r, \nu)=\sum_{i}\beta^{sca}_{i}(r, \nu), &&
\beta^{sca}_{i}(r, \nu)=n_{i}(r) \sigma_{i}^{sca}(\nu)
\label{eq:sca}
\end{eqnarray}
\begin{eqnarray}
\beta^{ext}(r, \nu)=\beta^{abs}(r, \nu)+\beta^{sca}(r, \nu)
\nonumber
\end{eqnarray}
where $n_{i}(r)$ is the density of the atmospheric molecule species $i$ and 
$\sigma_{i}^{abs}(\nu)$ and $\sigma_{i}^{sca}(\nu)$  are the
corresponding absorption and scattering cross sections. The phase function
is defined as
\begin{eqnarray}
p(r, \theta, \phi; \theta ', \phi ', \nu)=
\frac{\sum_{i}\beta_{i}^{sca}(r, \nu)
p_{i}(\theta, \phi; \theta ', \phi ', \nu)}
{\sum_{i}\beta_{i}^{sca}(r, \nu)}
\nonumber
\end{eqnarray}
where the phase function for each species 
\begin{eqnarray}
p_{i}(\theta, \phi; \theta ', \phi ', \nu)=
p_{i}(\cos\Theta, \nu)=
\frac{\sigma_{i}^{sca}(\nu, \cos\Theta)}
{\int_{4\pi}d\Omega\;\sigma_{i}^{sca}(\nu, \cos\Theta)}
\nonumber
\end{eqnarray}
and the scattering angle $\Theta$ is related to the local polar and 
azimuth angles through
\begin{eqnarray}
\cos\Theta = \cos\theta\;\cos\theta ' + \sin\theta\;\sin\theta '\;
\cos(\phi-\phi ').
\nonumber
\end{eqnarray}
The temperature profile, the densities and absorption and scattering 
cross sections are all needed to solve the radiative transfer equation.
Temperatures and densities may readily be obtained from measurements
or atmospheric models. Cross sections are taken from
measurements, from theoretical models or a combination of both.

\subsection{The radiative transfer equation in 1D}
In plane-parallel geometry the monochromatic\footnote{Frequency
  redistribution is required if Raman scattering is included in the
  calculation. For many applications Raman scattering is negligible
  and the photons are assumed not to change frequency. They are
  monochromatic. Thus, all frequency dependence have been suppressed
  in Eq.~\ref{eq:RTE_1D}.}
radiative transfer
equation~\ref{eq:Boltzmann_rte} is written by combining
Eq.~\ref{eq:PlaneParallelStreamingTerm} and
Eq.~\ref{eq:GeneralSource}
\begin{eqnarray}
 - \mu {dI(z,\mu,\phi) \over \beta^{ext}dz}&=& I(z,\mu,\phi) \nonumber\\
 && - {\omega(z) \over 4\pi } \int^{2\pi}_{0}d\phi' \int^{1}_{-1}d\mu'
   p(z,\mu,\phi;\mu',\phi')I(z,\mu',\phi') \nonumber\\
& &-(1-\omega(z))B[T(z)]
  \label{eq:RTE_1D}
\end{eqnarray}
where the single scattering albedo 
\begin{eqnarray}
\omega(z)=\omega(z,\nu)=
\frac{\beta_{i}^{sca}(z, \nu)}{\beta_{i}^{ext}(z, \nu)}=
\frac{\beta_{i}^{sca}(z, \nu)}{\beta_{i}^{abs}(z, \nu)+\beta_{i}^{sca}(z, \nu)}.
\nonumber
\end{eqnarray}
Formally the pseudo-spherical radiative transfer equation is similar
to Eq.~\ref{eq:RTE_1D}, but with $z$ replaced by $r$.



\subsection{Polarization - scalar versus vector}
The intensity or radiance $I$, solved for in the above equations have
a magnitude, a direction and a wavelength. In addition to this light
also possesses a property called polarization. When assuming randomly
oriented particles the radiative transfer equation formally does not
change when including polarization. However, the scalar radiance $I$
is replaced with the vector quantity ${\bf I}$
\begin{eqnarray}
{\bf I} = (I, Q, U, V),
\end{eqnarray}
where $I$, $Q$, $U$ and $V$ are the so-called Stokes parameters (see
e.g. \citet{bohren1998}). Furthermore, the phase function $p(r,\theta,
\phi; \theta', \phi')$ is replaced by the $4\times 4$ phase matrix ${\bf
  P}(r,\theta, \phi; \theta', \phi')$, and if thermal radiation is under
consideration the Stokes emission vector must also be accounted for.

The degree of polarization $p$ is defined as
\begin{equation}
  \label{eq:polarization:pol_degree}
  p = \frac{\sqrt{Q^2 + U^2 + V^2}}{I}.
\end{equation}
For completely polarized radiation, $Q^2 + U^2 + V^2 = I^2$, thus $p =
1$, and for unpolarized radiation, $Q = U = V = 0$, thus $p = 0$.

In addition to the degree of polarization, $p$, the degree of linear
polarization is defined as
\begin{equation}
  \label{eq:polarization:p_lin}
 p_{lin} = \frac{\sqrt{Q^2 + U^2}}{I},
\end{equation}
and the the degree of circular polarization is defined as
\begin{equation}
  \label{eq:polarization:p_circ}
 p_{circ} = \frac{V}{I}. 
\end{equation}
Polarization is often ignored in radiative transfer calculations both
due to the complexity involved in the solution of the RTE including
polarization and the higher demand on computer resources by these
solution methods. Also, for many applications polarization may be
ignored. If you are concerned about your specific application,
\code{uvspec} makes it easy to change solvers and thus readily allows 
comparisons to be made between scalar and vector calculations.


\section{General solution considerations}
A multitude of methods exist to solve the radiative transfer
equation~\ref{eq:Boltzmann_rte}. Most methods have some commonalities
and they are briefly described below.

\subsection{Direct beam/diffuse radiation splitting}
The integro-differential radiative transfer
equation~\ref{eq:Boltzmann_rte} gives the radiance field when solved
with  appropriate boundary conditions, that is, the 
radiation incident at the bottom and the top of the atmosphere. At the 
bottom of the atmosphere the Earth partly reflects radiation and also
emits radiation as a quasi-black-body. At the top of the atmosphere
($z=z_{{\text{toa}}}$) a parallel beam of  sunlight with magnitude
$I^{0}$ in the direction $\mu_{0}$ may be present
\begin{eqnarray}
I(z_{{\text toa}},\mu)= I^{0} \delta(\mu-\mu_{0}),
  \label{eq:Boundary_toa}
\end{eqnarray}
where $\delta(\mu-\mu_{0})$ is the Dirac delta-function. It is akward
to use a delta function for a boundary condition. However, a
homogeneous differential equation with inhomogeneous boundary conditions
may always be turned into an inhomogeneous differential equation with
homogeneous boundary conditions. Since the  integro-differential
equation~\ref{eq:Boltzmann_rte} is already inhomogeneous, the addition
of another inhomogeneous term does not necessarily complicate the problem.
Hence  the intensity field is written as the sum of the direct
($\text{dir}$) and the scattered ($\text{sca}$)(or diffuse) radiation
\begin{eqnarray}
I(z,\mu, \phi)=I^{\text{dir}}(z,\mu_0, \phi_0)+I^{\text{sca}}(z,\mu, \phi),
\label{eq:DirectDiffuse}
\end{eqnarray}
where $\mu_0$ and $\phi_0$ are the solar zenith and azimuth angles
respectively. Inserting Eq.~\ref{eq:DirectDiffuse} into
Eq.~\ref{eq:Boltzmann_rte} it is seen that the direct beam satisfies
\begin{eqnarray}
-\mu\frac{dI^{\text{dir}}(z,\mu_0, \phi_0)}{\beta^{ext}dz}=
-\mu\frac{dI^{\text{dir}}(z,\mu_0, \phi_0)}{d\tau}=I^{\text{dir}}(z,\mu_0, \phi_0)
  \label{eq:DirectRTE}
\end{eqnarray}
where the optical depth is defined as $d\tau = \beta^{ext}dz$.
The scattered intensity satisfies in 1D (the $\text{sca}$
superscript is omitted) 
\begin{eqnarray}
 - \mu {dI(\tau,\mu, \phi) \over d\tau}& =& I(\tau,\mu, \phi) 
 \nonumber \\ & & -
  {\omega(r) \over 4\pi } \int^{2\pi}_{0}d\phi' \int^{1}_{-1}d\mu'
   p(\tau,\mu, \phi;\mu', \phi')I(\tau,\mu', \phi)
 \nonumber \\ & & 
   -(1-\omega(\tau))B[T(\tau)] 
\nonumber \\
& &    -\frac{\omega(\tau)I^{0}}{4\pi}p(\tau,\mu, \phi;\mu_{0}, \phi_0 )e^{-\tau/\mu_{0}}.
  \label{eq:DiffuseRTE}
\end{eqnarray}
Solution of Eq.~\ref{eq:DirectRTE} for the direct beam yields the 
Beer-Lambert-Bouguer law
\begin{eqnarray}
I^{dir}(\tau,\mu_0)=I^{0}\;e^{-\tau/\mu_{0}}.
  \label{eq:Beerlaw}
\end{eqnarray}
The popular {\bf disort} solver \citep{Stamnes1988c,stamnes2000}
solves Eqs.~\ref{eq:DirectRTE}-\ref{eq:DiffuseRTE}.

\subsection{Pseudo-spherical approximation}
In the pseudo-spherical approximation the extinction path
$\tau/\mu_{0}$ in Eqs.~\ref{eq:DiffuseRTE} and \ref{eq:Beerlaw} is
replaced by the Chapman function, $ch(r,\mu_{0})$
\citep{Rees1989B,Dahlback1991}
\begin{eqnarray}
ch(r_{0},\mu_{0})=\int_{r_{0}}^{\infty}
              \frac{\beta^{ext}(r,\nu)\;dr}
        {\sqrt{1-\left(\frac{R+r_{0}}{R+r}\right)^{2}
         \left(1-\mu^{2}_{0}\right)}}.
  \label{eq2.24}
\end{eqnarray}
Here $R$ is the radius of the earth and $r_{0}$ the distance above the
earth's surface. The Chapman function describes the extinction
path in a spherical atmosphere.

Thus, in the pseudo-spherical approximation the direct beam is
correctly described by
\begin{eqnarray}
I^{dir}(r,\mu)=I^{0}\;e^{-ch(r,\mu_{0})}
  \label{eq:pseudoBeer}
\end{eqnarray}
and the diffuse radiation is approximated by replacing the
plane-parallel direct beam source in Eq.~\ref{eq:DiffuseRTE} with the
corresponding direct beam source in spherical geometry
\begin{eqnarray}
 - \mu {dI(\tau,\mu, \phi) \over d\tau}& =& I(\tau,\mu, \phi) 
 \nonumber \\ & & -
  {\omega(r) \over 4\pi } \int^{2\pi}_{0}d\phi' \int^{1}_{-1}d\mu'
   p(\tau,\mu, \phi;\mu', \phi')I(\tau,\mu', \phi)
 \nonumber \\ & & 
   -(1-\omega(\tau))B[T(\tau)] 
\nonumber \\
& &    -\frac{\omega(\tau)I^{0}}{4\pi}p(\tau,\mu, \phi;\mu_{0}, \phi_0 )e^{-ch(\tau,\mu_{0})}.
  \label{eq:pseudoDiffuseRTE}
\end{eqnarray}
The {\bf sdisort} solver included in the libRadtran software package
\citep{mayer2005} solves
Eqs.~\ref{eq:pseudoBeer}-\ref{eq:pseudoDiffuseRTE}. 

\subsection{Boundary conditions}
The diffuse radiative transfer Eq.~\ref{eq:DiffuseRTE} is solved
subject to boundary conditions at the top and bottom of the
atmosphere. At the top boundary there is no incident diffuse
intensity\footnote{The DISORT type RTE-solvers, {\bf disort 1.3}, {\bf
    disort 2.0}, {\bf sdisort} and {\bf twostr}, may include a diffuse
  radiation source at the top boundary. This may be of interest when
  for example modelling the aurora.}
($\mu \ge 0$) 
\begin{eqnarray}
I(\tau=0, -\mu, \phi )=0.
  \label{eq:TopBoundary}
\end{eqnarray}
The bottom boundary condition may quite generally be formulated in
terms of a bidirectional reflectivity, $\rho(\mu,\phi;-\mu',\phi')$,
and directional emissivity, $\epsilon(\mu)$,
\begin{eqnarray}
  I(\tau=\tau_g, \mu, \phi ) &=& \epsilon(\mu) B[T(\tau_{g})]
  + \frac{1}{\pi} \mu_0 I_0 e^{-\tau_g/\mu_{0}}
  \rho(\mu,\phi;-\mu',\phi') \nonumber \\
  && +\frac{1}{\pi} \int_0^{2\pi} d\phi' \int_0^1
  \rho(\mu,\phi;-\mu',\phi')  I(\tau,-\mu', \phi') \mu' d\mu' ,
  \label{eq:BottomBoundary}
\end{eqnarray}
where $T(\tau_{g})$ is the temperature of the bottom boundary, here
the Earth's surface.

In the case of a Lambertian reflecting bottom boundary with albedo
$\rho(\mu,\phi;-\mu',\phi')=A$,
Eq.~\ref{eq:BottomBoundary} simplifies to
\begin{eqnarray}
\pi I(\tau_{L}, \mu )&=& \pi\;\epsilon \;B[T(\tau_{g})]+
\mu_{0}\;A\;I^{0}e^{-\tau_{g}/\mu_{0})} \nonumber \\
&&+ 2\pi\; A \int_0^{2\pi}
d\phi' \int_{0}^{1} \mu I(\tau_{L},-\mu, \phi) d\mu.
  \label{eq:BottomBoundarySimple}
\end{eqnarray}
The albedo, $A$, gives the fraction of reflected light under the
assumption that the surface reflects radiation isotropically (Lambert
reflector). The emissivity $\epsilon = 1-A$, by Kirchhoff's law.
In both Eqs.~\ref{eq:BottomBoundary} and
\ref{eq:BottomBoundarySimple} the first term on the right hand side is
the thermal radiation emitted by 
the surface. The second term is due to reflection of the direct beam
that has penetrated through the whole atmosphere and the last term 
is reflection of downward diffuse radiation

\subsection{Separation of the azimuthal $\Phi$-dependence, Fourier decomposition}
For scattering processes in the atmosphere the scattering phase
function depends only on the angle $\Theta$ between the incident and
scattered beams. This may be used to seperate out the
$\Phi$-dependence in Eqs.~\ref{eq:DiffuseRTE} and
\ref{eq:pseudoDiffuseRTE} as follows. The phase function is first
expanded as a series of Legendre polynomials
\begin{eqnarray}
  p(\tau,\mu, \phi;\mu', \phi') = p(\tau,\Phi) = \sum_{l=0}^{2M-1}
  (2l+1) g_l(\tau) p_l(\cos\Phi)
  \label{eq:PhaseFunction}
\end{eqnarray}
where the phase function moments $g_l$ are given by
\begin{eqnarray}
  g_l(\tau) = {1 \over 2} \int_{-1}^{+1} p_l(\cos\Phi)p(\tau,\Phi)d(\cos\Phi).
  \label{eq:PhaseFunctionMoment}
\end{eqnarray}
The $g_1$ term is called the ``asymmetry factor'', and $g_0=1$ due to
normalization of the phase function. Applying the addition theorem for
spherical harmonics to Eq.~\ref{eq:PhaseFunction} gives
\begin{eqnarray}
p(\tau,\Phi) = \sum_{l=0}^{2M-1}  (2l+1) g_l(\tau)\left\{
    p_l(\mu) p_l(\mu') + 2\sum_{m=1}^{l} \Lambda_l^m(\mu)
    \Lambda_l^m(\mu') \cos m(\phi-\phi') \right\}\nonumber \\
  \label{eq:PhaseFunctionSphericalHarmonics}
\end{eqnarray}
where the normalized associated Legendre polynomials are defined as
\begin{eqnarray}
  \Lambda_l^m(\mu) = \sqrt{\frac{(l-m)!}{(l+m)!}} P_l^m(\mu),
  \label{eq:NormalizedAssociatedLegendrePolynomial}
\end{eqnarray}
and $P_l^m(\mu)$ are the standard Legendre polynomials. The cosine
dependence of the phase function,
Eq.~\ref{eq:PhaseFunctionSphericalHarmonics}, suggests that cosine
expansion of the intensity may be fruitful. Expanding the intensity as
a cosine Fourier series:
\begin{eqnarray}
  I(\tau,\mu,\phi) = \sum_{l=0}^{2M-1} I^m(\tau,\mu)  \cos m(\phi_0-\phi)
  \label{eq:CosineExpansion}
\end{eqnarray}
and inserting into Eqs.~\ref{eq:DiffuseRTE} and
\ref{eq:pseudoDiffuseRTE} gives $2M$ independent integro-differential
equation (only the plane-parallel version is shown here)
\begin{eqnarray}
 - \mu {dI^m(\tau,\mu) \over d\tau}& =& I^m(\tau,\mu) 
 \nonumber \\ & & -
  {\omega(r) \over 2 } \int^{1}_{-1}d\mu'
   \sum_{l=m}^{2M-1} (2l+1) g_l(\tau) \Lambda_l^m(\mu) \Lambda_l^m(\mu') I^m(\tau,\mu')
 \nonumber \\ & & 
   -\delta_{m0}(1-\omega(\tau))B[T(\tau)] 
\nonumber \\
& &    -\frac{\omega(\tau)I^{0}}{4\pi} (2-\delta_{m0}) 
        \sum_{l=m}^{2M-1} (2l+1) g_l(\tau) \Lambda_l^m(\mu) \Lambda_l^m(\mu') e^{-\tau/\mu_{0}}.\nonumber\\
  \label{eq:DiffuseRTEseparated}
\end{eqnarray}
where 
\begin{displaymath}
\delta_{m0} = \left\{ \begin{array}{ll}
      1 & \text{if $m=0$} \\
      0 & \text{if $m\neq 0$} 
      \end{array} \right.
\end{displaymath}



\subsection{Calculated quantities}
Solution of the radiative transfer equation generally yields the
diffuse radiance
\begin{eqnarray}
  I(\tau, \mu, \phi)
  \label{eq:Radiance}
\end{eqnarray}
and the direct radiance
\begin{eqnarray}
  I^{\text dir}(\tau, \mu_0, \phi_0).
  \label{eq:DirectRadiance}
\end{eqnarray}
For the solvers that include polarization the vector quantities of the
above quantities are calculated.
From these quantities the upward, $E^{+}(\tau)$, and downward,
$E^-(\tau)$, fluxes, or irradiances, are calculated
\begin{eqnarray}
E^+(\tau) &=& \int_0^{2\pi} d\phi \int_{0}^{1} \mu I(\tau,\mu, \phi) d\mu 
  \label{eq:UpwardFlux} \\
E^-(\tau) &=& \mu_0 I_0 e^{-\tau/\mu_0} + \int_0^{2\pi} d\phi \int_{0}^{1} \mu I(\tau,-\mu, \phi) d\mu.
  \label{eq:DownwardFlux}
\end{eqnarray}
Furthermore, the mean intensity 
\begin{eqnarray}
  \overline{I}(\tau) &=& \frac{1}{2\pi} \left[ I_0 e^{-\tau/\mu_0} +
 \int_0^{2\pi}d\phi \int_{0}^{1} I(\tau,-\mu, \phi) d\mu + 
 \int_0^{2\pi}d\phi \int_{0}^{1}  I(\tau,\mu, \phi) d\mu \right],\nonumber\\
  \label{eq:MeanIntensity} 
\end{eqnarray}
is related to the actinic flux
\citep{Madronich1987b}, $F$, used for the calculation of photolysis
(or photodissociation) rates
\begin{eqnarray}
  F(\tau) &=& 4 \pi \overline{I}(\tau).
  \label{eq:ActinicFlux} 
\end{eqnarray}
Finally, heating rates may be calculated from either the flux
differences or the mean intensity.
\begin{eqnarray}
  \frac{\partial T}{\partial t} &=& -\frac{4\pi}{c_p \rho_m}
  \frac{\partial E}{\partial z} =  -\frac{4\pi}{c_p \rho_m}
  (1-w)(\overline{I}-B)\frac{\partial \tau}{\partial z}.
  \label{eq:HeatingRate} 
\end{eqnarray}
Note that the partial derivative of $\tau$ with respect to $z$ is needed
since optical properties and $\overline{I}$ are calculated as functions
of $\tau$.

The various radiative transfer equation solvers included in the
\code{uvspec} tools in the {\sl libRadtran} package, have different
capabilities to calculate the above 
radiative quantities. The user is referred to
section~\ref{sec:RTEsolvers} for an overview of the different
solvers included in the \code{uvspec} program and their respective
capabilities. For a complete description of all solvers with options
section~\ref{sec:uvspec_options} should be consulted. Finally, there
is nothing to complement a thorough understanding of the problem at
hand, the theory behind the chosen solution and a little reading of
the code itself. 

\subsection{Lidar equation}

The lidar equation can be written as (see e.g.~\citet{Weitkamp2005})
%
\begin{equation}
%
\frac{{\rm d}N(r)}{{\rm d}r} =
  \frac{E_0}{E_{\rm phot}} A_{\rm det} \eta \frac{O(r)}{4 \pi r^2}
  p(\cos \pi) \beta^{sca}(r)
  \exp\left(-2\int_0^r dr' \beta^{ext}(r')\right),
%
\label{eq:sslidar}
\end{equation}
%
where $N(r)$ is the number of detected photons, $E_0$ is the energy
per laser pulse, $E_{\rm phot}$ is the energy per photon, $A_{\rm
  det}$ is the detector area, $\eta$ is the detector efficiency,
$O(r)$ is the overlap function, $r$ is the range, and $p(\cos \pi)$ is
the scattering phase function in backward direction. Note that the
nomenclature here is consistent with the libRadtran documentation and
differs from that in most lidar papers and books.

The lidar equation is a solution of the RTE for the special problem of
a lidar signal, and is a single scattering approximation to the real
world. Nevertheless it is applicable for many cases of interest. For
space-borne lidars it should not be used.

Many lidarists are also interested in the lidar ratio, which is
defined as
%
\begin{equation}
%
S(r) = \frac{4 \pi}{p(\cos \pi) \omega(r)}.
%
\label{eq:lidar_ratio}
\end{equation}
%

For the special case of Lambertian surface reflection, the signal is
%
\begin{equation}
%
N_{\rm surf}(r_{\rm surf}) =
  \frac{E_0}{E_{\rm phot}} A_{\rm det} \eta \frac{O(r_{\rm surf})}{4 \pi r_{\rm surf}^2}
  4 a \cos \theta_{\rm refl}
  \exp\left(-2\int_0^{r_{\rm surf}} dr' \beta^{ext}(r')\right),
%
\label{eq:sslidar_surf}
\end{equation}
%
where $r_{\rm surf}$ is the range of the surface, $a$ is the surface
albedo, and $\theta_{\rm refl}$ is the inclination with which the
laser beam hits the surface.

\subsection{Verification of solution methods}
To solve the radiative transfer equation involves complex numerical
procedures that are difficult both to develop and to implement. Great
care must be taken during implementation to assure that the numerical
procedure is stable for any values and combinations of 
the input parameters, i.e. optical depth, single scattering albedo, phase
function and boundary conditions. The testing of new solvers are
typically done by the developers against analytical solutions which
are available for a few special cases. Furthermore, tests and
comparisons are made against other models and measurements. The reader
is referred to the individual papers describing the various solvers
for more information.

The input quantities needed by the solvers are optical depth, single
scattering albedo, phase function and boundary conditions. These are
calculated from atmospheric profiles of molecular density, trace gas
species, water and ice clouds and aerosols. In addition,
the absorption and scattering properties of the various species are
taken from measurements or model calculations. The calculation of the
optical properties are compared against other models and measurements
during code development.


